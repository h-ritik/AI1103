\documentclass{article}

\usepackage{array}
\usepackage{longtable}
\usepackage{units}
\usepackage{booktabs}
\usepackage{graphicx}
\usepackage{amsmath, amsthm, amssymb, bm}
\usepackage{tikz, pgfplots}
\usepackage{lipsum}
\usepackage{mwe}
\usetikzlibrary{shapes, arrows, positioning, fit, calc}
\newtheorem{theorem}{Theorem}
\newtheorem{Property}{Property}
\theoremstyle{remark}
\newtheorem*{defn}{Definition}
\renewcommand{\vec}[1]{\underline{#1}}
\usepackage{units}
\usepackage{fancyvrb}
\usepackage{hyperref}
\fvset{fontsize=\normalsize}

\title{AI1103 Assignement 5}
\author{Hritik Sarkar}

\newcommand\numberthis{\addtocounter{equation}{1}\tag{\theequation}}
\newcommand\inv[1]{#1\raisebox{1.15ex}{$\scriptscriptstyle-\!1$}}

\begin{document}
\maketitle
Paper - \href{https://github.com/gadepall/papers/blob/master/ugc/math/dec-math-set-a-2018.pdf}{UGC 2018-Dec}
\newline
\newline
Q60. In an airport, domestic passengers and international passengers arrive independently according to Poisson processes with rates 100 and 70 per hour. If it is given that the total no. of passengers (domestic and international) arriving in that airport between 9:00 AM and 11:00 AM on a particular day was 520, then what is the conditional distribution of the no. of domestic passengers arriving in this period ?
\newline
\\
1. Poisson (200)\\ \\
2. Poisson (100)\\ \\
3. Binomial (520,$\frac{10}{17}$)\\ \\
4. Binomial (520,$\frac{7}{17}$)\\ \\
Answer. Let $X_1$ and $X_2$ be the random variables representing the no. of arrivals of domestic and international passengers during the time interval 9:00 AM to 11:00 AM. So,
\[
    X_1\,\sim\,Poisson(\lambda_1 = 200),\; \; X_2\,\sim\,Poisson(\lambda_2=140)
\]
And we have to find (let n = 520),
\[
    P(X_1 = x | X_1+X_2 = n) 
\]
Simplifying,
\begin{align*}
    P(X_1 = x | X_1+X_2 = n) &= \frac {P(X_1 = x, X_1+X_2 = n)} {P(X_1+X_2)=n}\\
    &= \frac {P(X_1 = x, X_2 = n-x)}{P(X_1+X_2)=n}\\
    &= \frac{P(X_1=x)\:P(X_2=n-x)}{P(X_1+X_2=n)} \numberthis{\label{1}}&\textit{($X_1,X_2$ are independent)}
\end{align*}
Now, the denominator is a summation of two independent Poisson random variables. We will prove that, this summation is also a Poisson random variable.\\ \\
Moment Generating Function of $X \sim Poisson(\lambda)$ is given by,
\begin{align*}
    M_X(t) &= \mathbb{E}\left [ e^{tX} \right ]\\
    &= \sum_{i=0}^{\infty} \frac{{}e^{-\lambda}\lambda^ie^{ti}}{i!}\\
    &= e^{-\lambda}\sum_{i=0}^{\infty} \frac{{}(e^t\lambda)^{i}}{i!}\\
    &= e^{-\lambda}e^{e^t\lambda}\\
    &= e^{e^t\lambda-\lambda}
\end{align*}
So,
\[
    M_{X_1}(t)= e^{e^t\lambda_1-\lambda_1} \; \; M_{X_2}(t)= e^{e^t\lambda_2-\lambda_2}
\]
\begin{align*}
    M_{X_1+X_2}(t) &= \mathbb{E} \left [ e^{t(X_1+X_2)} \right ]\\
    &= \mathbb{E} \left [ e^{tX_1} e^{tX_2} \right ]\\
    &= \mathbb{E} \left [ e^{tX_1} \right ] \mathbb{E} \left [ e^{tX_2} \right ] &\textit{(independence)}\\
    &= e^{e^t\lambda_1-\lambda_1} e^{e^t\lambda_2-\lambda_2}\\
    &= e^{e^t(\lambda_1+\lambda_2)-(\lambda_1+\lambda_2)}
\end{align*}
MGF of $X_1+X_2$ is same as that of a Poisson random variable with rate $\lambda_1+\lambda_2$.\\ \\
Now, going back to equation \eqref{1}
\begin{align*}
    P(X_1 = x | X_1+X_2 = n) &= \frac{P(X_1=x)\:P(X_2=n-x)}{P(X_1+X_2=n)} \\
    &= \frac{e^{-\lambda_1}{\lambda_1^x}}{x!} \times \frac{e^{-\lambda_2}{\lambda_2^{n-x}}}{(n-x)!} \times \frac{n!}{e^{-(\lambda_1+\lambda_2)}{(\lambda_1+\lambda_2)^n}} \\
    &= \frac{n!}{x!\;\;n-x!} \times \left ( \frac{\lambda_1}{\lambda_1+\lambda_2} \right )^x \times \left ( \frac{\lambda_2}{\lambda_1+\lambda_2} \right )^{n-x}
\end{align*}
We can see that the distribution is binomial with n = 520 and p = $\frac{10}{17}$. Thus option 3 is correct. (Answer)
\end{document}